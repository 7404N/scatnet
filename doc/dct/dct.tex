\documentclass[twocolumn]{article}
\usepackage{amsmath,cite}
\usepackage{graphicx}

\usepackage{amssymb}

\usepackage{url}

\usepackage[framed,numbered,autolinebreaks,useliterate]{../common/mcode}

\graphicspath{{figures/}}

\title{DCT filtering}

\begin{document}
	
\maketitle

\section{1D case}

\subsection{Decomposing of $x$}

Given a signal $x[n]$, defined for $n = 0,\ldots,N-1$, we can define the discrete cosine transform $Cx[k]$ as
\[
Cx[k] = 2\sum_{n=0}^{N-1}x(n)\cos\left(\frac{2\pi\left(n+\frac{1}{2}\right)k}{2N}\right)
\]
Note that this corresponds to an unnormalized version of the DCT calculated by  the \mcode{dct} function in MATLAB. From $x$, we can define the symmetrized signal $y[n]$ by
\[
y[n] = \left\{\begin{array}{ll} x[n], & n=0,\ldots,N-1 \\ x(2N-n-1), & n = N,\ldots,2N-1\end{array}\right.,
\]
and its discrete Fourier transform $\hat{y}[k]$
\[
\hat{y}[k] = \sum_{n=0}^{2N-1}y[n]e^{\frac{-2\pi i n k}{2N}}.
\]
One can then derive the following relation between $\hat{y}$ and $Cx$ \cite{makhoul}:
\[
\hat{y}[k] = e^{\frac{2\pi i k/2}{2N}}Cx[k],\quad k=0,\ldots,N-1.
\]
Note that since the signal $y[n]$ is symmetric not about $n=0$, but $n=-1/2$, its Fourier transform is not real, but has a linear phase corresponding to this shift of symmetry axis. We also note that since $y$ is real, the remaining coefficients of $\hat{y}$ can be obtained through conjugate (Hermitian) symmetry
\[
\hat{y}[2N-k] = \overline{\hat{y}[k]},\quad k=1,\ldots,N-1
\]
and the fact that $\hat{y}[N] = 0$.

\subsection{Filtering with a Hermitian filter}

If we want to calculate the circular convolution of $x$ with a filter $h$ using symmetric boundary conditions, this can be obtained by filtering $y$ with $h$ using periodic boundary conditions, which is thus reduced to multiplication of Fourier transforms. For a general $h$, this breaks the symmetry of $y$ and required to calculate the convolution for all $n=0,\ldots,2N-1$. However, if $h$ possesses symmetry properties, for example conjugate symmetry about $n = 0$, we only need to compute transforms of size $N$.

Let $h$ be defined on $n=0,\ldots,2N-1$ with conjugate symmetry about $n=0$ (even symmetry in the real part and odd symmetry in the imaginary part). Its Fourier transform $\hat{h}$ is therefore real. Calculating the product
\[
\hat{z}[k] = \hat{y}[k]\hat{h}[k]
\]
the phase remains unchanged, so the signal $z$ whose Fourier transform is $\hat{z}$ has a conjugate symmetry about $n=-1/2$. Note that in extending $Cx[k]$ into $\hat{y}[k]$, we've multiplied the number of real coefficients by $2$ (disregarding the phase, which is fixed). However, since $h$ is a complex filter as opposed to purely real, it is unavoidable that the number of coefficients would double at some point.

We can exploit the fact that $\hat{z}[k]$ is real times a linear phase, reducing its inversion from a complex inverse Fourier transform of size $2N$ to one of size $N$ \cite{brigham}.

Let us divide $\hat{z}$ into its even-numbered coefficients $\hat{z}_0$ and its odd-numbered coefficients $\hat{z}_1$ (shifted in phase):
\begin{align*}
\hat{z}_0[k] &= \hat{z}[2k] & k=0,\ldots,N-1 \\
\hat{z}_1[k] &= e^{\frac{-2\pi i/2}{2N}}\hat{z}[2k+1] & k=0,\ldots,N-1
\end{align*}
We can now create a signal $\hat{s}[k]$, defined by
\[
\hat{s}[k] = \hat{z}_0[k]+i\hat{z}_1[k].
\]
Here, $\hat{s}[k]$ is a complex signal of length $N$. If we compute its inverse Fourier transform, we obtain
\[
s[n] = z_0[n]+iz_1[n],
\]
where $z_0$ and $z_1$ are the inverse Fourier transforms of $\hat{z}_0$ and $\hat{z}_1$, respectively. Since $\hat{z}_0[k]$ and $\hat{z}_1[k]$ both have the phase $e^{\frac{2\pi i k/2}{N}}$, their inverses $z_0[k]$ and $z_1[k]$ are conjugate symmetric about $n=-1/2$. As a result, the real even and the imaginary odd parts of $s$ belong to $z_0$ while the real odd and the imaginary even parts of $s$ belong to $z_1$. Specifically, if we define
\begin{align*}
s_e[n] &= \frac{1}{2}\left(s[n]+s[N-1/2-n]\right) \\
s_o[n] &= \frac{1}{2}\left(s[n]-s[N-1/2-n]\right),
\end{align*}
we obtain
\begin{align*}
z_0[n] &= \mathrm{Re}\{s_e\}+i\mathrm{Im}\{s_o\}
z_1[n] &= \mathrm{Im}\{s_e\}-i\mathrm{Re}\{s_o\}.
\end{align*}
Inverting the phase shift of $z_1$, we can now reconstruct $z$:
\[
z[n] = z_0[n]+e^{2\pi i n/2N}\left(e^{-\frac{2\pi i/2}{2N}}z_1[n]\right),
\]
for $n=0,\ldots,N-1$. Since $z$ is conjugate symmetric about $n=-1/2$, we don't need to calculate the second half.

\subsection{Filtering with an even or odd filter}

TODO! Instead of considering an analytic Hermitian filter, we can consider two (orthogonal) filters: the real and imaginary part. The real part is even, so can be decomposed with a DCT, while the imaginary part is odd, so can be decomposed with a DST. Computing the convolution of even and odd functions can be done by multiplying their DCTs/DSTs \cite{martucci}. The issue is then how to optimize the inversion so that the inverse DCT/DST can be computed without padding. To do this, we should be able to adapt the methods outlined in \cite{makhoul}.

\bibliographystyle{IEEEtran}
\bibliography{dct}

\end{document}
